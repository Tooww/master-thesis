
\section*{Project title \red{(Phase 2)}}
\textbf{$\sim$4 page}

This paragraph must describe briefly the general principle of the project, what it is, how it works and what outputs are expected. Comments if the results are different. 

\subsection*{Overview}
In this section, student must present the general project overview of the system with a brief description of the main components. 


\subsection*{Components}

The student present how are implemented the different components. This section must be very clear with schematics, calculations, models, list of equipment or resources, etc. 

The student creates a new \textit{subsection} for each component. Although the grading is slightly different between the Designer and Engineer, we encourage the students to give as many details and many subsections as possible (both from Engineer and Design). It is also possible to add other subsections. 

\textbf{Project components for Creative Technology Engineer}

Here is a list of subsection that and Engineer CreaTech must present (at least 4). 

\begin{itemize}
    \item System Architecture
    \item Simulation and / or Modelization
    \item Fabrication and / or Implementation
\end{itemize}


\textbf{Project components for Creative Technology Designer}

Here is a list of subsection that and Design. CreaTech must describe (at least 4). 


\begin{itemize}
    \item 3D mockup and / or Design 
    \item Ergonomy and / or Finish Quality
    \item Fabrication
\end{itemize}

\subsection*{Product Specific Success Criteria}

The student proposes a list of success criteria specific of the projects with the corresponding evaluation marks.

\textbf{Example of PSSC}
\begin{table}[h!]
    \begin{tabular}{|c|l|c|}
      \hline
      Items & Description & Mark (total 20) \\
      \hline
      Robot Navigation & The robot moves forward a straight line & 1 \\
      \cline{2-3} & The robot avoids obstacles  & 3 \\
      \cline{2-3} & The robot plans trajectory obstacles  & 2 \\
      \hline 
      Robot Arm & The robot arm is moving & 1 \\
      \hline 
      ... & ... & ... \\
      \hline 
      Total &  & 20 \\
      \hline
    \end{tabular}
\end{table}


\pagebreak