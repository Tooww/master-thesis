\chapter*{Preface}
\addcontentsline{toc}{chapter}{Preface} % Add the preface to the table of contents as a chapter

The De Vinci Innovation Center (DVIC) is a community of makers that develops
technologies within philosophical and critical frameworks to shape our societies’
futures. The objective is to implement real-world solutions as well as design projects
to enhance public engagement, improve education, and overall provide scientific
knowledge. Our researchers contribute actively to top-level international research
in multiple fields, including artificial intelligence, human-computer interactions,
education, and ecology. We believe that these objectives require a transdisciplinary
approach, that bridges the gap between sciences, techniques, sociology, and philosophy. This is performed by collaborating with other scientists and industrial and
startup sharing our values, to form strong research partnerships...


The Artificial Lives group, led by Dr. Clement Duhart, aims to develop the next generation of machines and Human-Machine Interfaces. The group members strongly
believe that through the combination of Design and Engineering, human-centered
technologies can blend into our environments to become invisible, vastly improving
daily lives. To achieve this vision, the members contribute to human-computer
interactions, cognitive enhancement through new forms of extended intelligence,
learning platforms, and cobotic. Our bio-inspired, multidisciplinary approach couples AI and virtual reality with intelligent materials, robotics and the Internet of
Things.


For the past two years, De Vinci Innovation Center (DVIC) students following
the Creative Technologies curriculum had the opportunity to develop their vision
on technology, innovation, and society. This proceeding is a composition of six
master’s theses, ranging from Machine Learning, Human-Computer-Interaction to
Robotics. The authors strongly believe that developing alternative futures requires
new types of engineering that take into consideration both the people’s needs and
the environment. These documents have been written to reflect this vision and
refined over several months with an iterative reviewing supervised by the Principal
Investigators.


The Authors, the Principal Investigators and the whole DVIC community is proud
of releasing this first proceeding. We dedicate this first edition to Pascal Brouaye
and Nelly Rouyres, without whom nothing would have been possible.

