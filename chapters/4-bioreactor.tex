\chapter{SCOBY Machine \& Materials}


\section{Overview}

In the same way as for the environent control project with the scoby, we had to think about how machines can help us make this new material our own, in much the same way as 3D printers popularized the use of PLA. 

The methods and machines presented here are designed to facilitate the use of this biomaterial, with the aim of optimizing the production of bacterial cellulose and changing the growing conditions to produce 3D cellulose. 

In addition, unlike mycelium, which unfolds in a substrate contained in a mold, cellulose requires more attention to shape channelling, both for molding and demolding. 

% \begin{figure}[h]
%     \centering
%     \includegraphics{images/Myceliummachine.png}
%     \caption{System design representation}
%     \label{fig:}
% \end{figure} 

\section{Systems design}

\section{Manufacturing Processes \& Grow Theory}

\section{Contribution}

\subsection{Rotary Bioreactor}


\section{Result}

