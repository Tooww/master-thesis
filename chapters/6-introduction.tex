\chapter{introduction market field}

\section{Context}

In the field of material science, Plastics have been booming since the late 1950s\cite{geyer2017production}.It's a problem for ecosystems, on one hand, by the means of production as 90\% of plastics come from fossil fuels. On other hand,because of its widespread presence in ecosystems gradually degraded into microplastics and nanoplastics. Moreover only 9\% of plastics is recycled and 12\% is incinerated almost all other waste is lost in nature\cite{natureeditorial}.

With plastics, we have a material that our environments can't dissolve and similarly as a a living organism that create it creates an imbalance. Since we have create materials that are "invincible" for the environment,  that environments can't gets rid of, or can't use we've created materials that pollute nature once they're out there.

Moreover we we're moving towards a problem of raw materials in general, independently of plastics. this raises other issues of recycling, delocalization of resources, and the energy needed to extract them from the environment.

That is why, some researcher and designer or other indistrual actor, works on the fabrication of new kind of materials that are biobased.
 
These biosourced materials, which we call biomaterials, have the particularity of being co-created (and sometimes even co-designed) with living organisms. As they are organic materials, this makes them eco-responsible for the environment. 

From an economic point of view, it's important to understand how industrial players emerge? how they are financed and with what economic model? how the biomaterials sector is received by other industrial players? how these new approaches are received by the general public?

because it's through these new companies that certain new solutions will emerge. Questions concerning legislation are also very important to take into account, since we are dealing with materials that can be manufactured from living organisms, certain questions arise concerning patents and, in some cases, bioethics laws concerning their commercialization.  

A good example to illustrate the importance of the law is how the electric car sector has been impacted by European legislation concerning the ban on internal combustion cars by 2035. 
In a sector closer to the one we're studying, the ban on plastic straws in France (except for medical use) and how this has boosted companies making straws from environmentally-friendly materials such as wheat, rice or pasta straws.

\section{Approach}



\section{Field of Research}



\section{Contributions}




\section{Problematic}