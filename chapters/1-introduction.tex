%\justifying
\chapter{Introduction}
\labch{intro}

\section{Context}

In the field of material science, Plastics have been booming since the late 1950s\cite{geyer2017production}.It's a problem for ecosystems, on one hand, by the means of production as 90\% of plastics come from fossil fuels. On other hand,because of its widespread presence in ecosystems gradually degraded into microplastics and nanoplastics. Moreover only 9\% of plastics is recycled and 12\% is incinerated almost all other waste is lost in nature\cite{natureeditorial}.

With plastics, we have a material that our environments can't dissolve and similarly as a a living organism that create it creates an imbalance. Since we have create materials that are "invincible" for the environment,  that environments can't gets rid of, or can't use we've created materials that pollute nature once they're out there.

Moreover we we're moving towards a problem of raw materials in general, independently of plastics. this raises other issues of recycling, delocalization of resources, and the energy needed to extract them from the environment.

That is why, some researcher and designer or other indistrual actor, works on the fabrication of new kind of materials that are biobased.
 
These biosourced materials, which we call biomaterials, have the particularity of being co-created (and sometimes even co-designed) with living organisms. As they are organic materials, this makes them eco-responsible for the environment. 

brev histoire biomat 

\section{Approach}

In the field of material science, Plastics have been booming since the late 1950s\cite{geyer2017production}.
It's a problem for ecosystems, on one hand, by the means of production as 90\% of plastics come from fossil fuels. On other hand,
because of its widespread presence in ecosystems gradually degraded into microplastics and nanoplastics.
Moreover only 9\% of plastics is recycled and 12\% is incinerated almost all other waste is lost in nature\cite{natureeditorial}.

With plastics, we have a material that our environments can't dissolve and similarly as a a living organism that create
it creates an imbalance. Since we have create materials that are "invincible" for the environment,  that environments can't gets rid of, or can't use
we've created materials that pollute nature once they're out there

Moreover we we're moving towards a problem of raw materials in general, independently of plastics. this raises other issues of recycling, 
delocalization of resources, and the energy needed to extract them from the environment.


\section{Field of Research}

In the field of material science, Plastics have been booming since the late 1950s\cite{geyer2017production}.
It's a problem for ecosystems, on one hand, by the means of production as 90\% of plastics come from fossil fuels. On other hand,
because of its widespread presence in ecosystems gradually degraded into microplastics and nanoplastics.
Moreover only 9\% of plastics is recycled and 12\% is incinerated almost all other waste is lost in nature\cite{natureeditorial}.

With plastics, we have a material that our environments can't dissolve and similarly as a a living organism that create
it creates an imbalance. Since we have create materials that are "invincible" for the environment,  that environments can't gets rid of, or can't use
we've created materials that pollute nature once they're out there

Moreover we we're moving towards a problem of raw materials in general, independently of plastics. this raises other issues of recycling, 
delocalization of resources, and the energy needed to extract them from the environment.

\section{Contributions}

In the field of material science, Plastics have been booming since the late 1950s\cite{geyer2017production}.
It's a problem for ecosystems, on one hand, by the means of production as 90\% of plastics come from fossil fuels. On other hand,
because of its widespread presence in ecosystems gradually degraded into microplastics and nanoplastics.
Moreover only 9\% of plastics is recycled and 12\% is incinerated almost all other waste is lost in nature\cite{natureeditorial}.

With plastics, we have a material that our environments can't dissolve and similarly as a a living organism that create
it creates an imbalance. Since we have create materials that are "invincible" for the environment,  that environments can't gets rid of, or can't use
we've created materials that pollute nature once they're out there

Moreover we we're moving towards a problem of raw materials in general, independently of plastics. this raises other issues of recycling, 
delocalization of resources, and the energy needed to extract them from the environment.
