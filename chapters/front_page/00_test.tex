%\setchapterpreamble[u]{\margintoc}
\chapter{Introduction}
\labch{intro}

\section{intro}
In the field of educational technology and human computer interaction, the distinction between learning
and performance forms a critical foundation for understanding how interfaces can enhance both aspects.
\section{intro2}
\section{intro3}
\subsection{into3.1?}


\chapter{Construction and use of Biomaterial}


\section{Biological Definition}

\subsection{S.C.O.B.Y}
\subsection{Mycelium}

\section{2D Cellulose}

\subsection{Growth Theory}
\subsection{Manufacturing Processes}


\section{3D Cellulose}

\subsection{Growth Theory}
\subsection{Manufacturing Processes}



\section{Mycelium Brick}


\subsection{Mycelium Brick}
\subsection{Manufacturing Processes}
\subsection{Mechanical Test}



\section{Discution \& Framework}


\chapter{Construction and use of Bioreactor}


\section{Definition}


\section{General Architecture}


\section{Rotating Bioreactor}

\subsection{Overview}
\subsection{System Design}
\subsection{Contribution}


\section{Mycelium Bioreactor}

\subsection{Overview}
\subsection{System Design}
\subsection{Contribution}

\section{Modular Bioreactor}

\subsection{Overview}
\subsection{System Design}
\subsection{Contribution}


