
% \chapter*{Useful resources: Good practises}

% \subsection{Documenting your work}

% It’s important to keep track of your work for 1) have all the resources to report it in your manuscript, and 2) be able to share it with others to build on it.  We recommend taking pictures regularly of the prototypes you build, keep notes of meetings with dates to be able to trace back your thoughts and keep various versions of your work to be able to compare them later on.

% When coding, a version control software like git is a powerful tool that you should use extensively. Ask for advice in the lab about good usages.


% \subsection{Finding and Organizing Related Work}

% During the seminar and the thesis, you will read a considerable amount of papers. Searching for articles can be done in various ways. A good starting point is to use Google Scholar or the ACM Digital Library.

% To avoid reading the papers multiple times, it is recommended to keep track of all references and take notes. Free applications such as Zotero or Mendeley facilitate managing references and keeping notes for each of them. We highly recommend using such a tool, as it can drastically enhance your efficiency when reviewing the literature.


% \subsection{Framing and Context}

% To explain the context, you should address the following questions. 
% \begin{itemize}
%     \item What is the context of this project?
%     \item What is the selected approach?
%     \item Why is it important?
%     \item Why is this task hard? Is this project more than an engineering effort? If not, why is the challenge interesting?
%     \item  How does your intended contribution relate to the state of the art?
% \end{itemize}


% \subsection{Making a plan}

% Once you have a defined storytelling and narration,  you should, with the help of your advisor, define a plan how to present it. For that, consider the following questions for each project. 

% \begin{itemize}
%     \item  What are the steps needed to achieve the expected contribution?
%     \item  What would be the alternatives? Why did you choose this particular one?
%     \item  What are the risks? what are the strategies to mitigate those risks?
%     \item  How are you going to validate your outcomes?
% \end{itemize}




% % \input{chapters/chapter01}
% % \input{chapters/chapter02}
% % \input{chapters/chapter03}
% % \input{chapters/chapter04}
% % \input{chapters/chapter05}

% % \pagelayout{wide} % No margins
% % \addpart{Design and Additional Features}
% % \pagelayout{margin} % Restore margins
